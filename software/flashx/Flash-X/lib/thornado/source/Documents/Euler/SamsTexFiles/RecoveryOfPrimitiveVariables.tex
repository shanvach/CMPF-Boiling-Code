\section{Recovery of Primitive Variables}
In order to recover the primitive from the conserved variables we need to solve the nonlinear equation:
\begin{equation}
    f\left(p\right)=p-\ol{p}\left(p\right)=0,
\end{equation}
where $\ol{p}\left(p\right)$ is the pressure as obtained via the ideal gas equation of state with an initial guess, $p$:
\begin{equation}
    \ol{p}=\left(\Gamma-1\right)\rho\,\epsilon,
\end{equation}
where
\begin{equation}
    \rho=\rho\left(\bU,p\right),\hspace{1em}\epsilon=\epsilon\left(\bU,p\right).
\end{equation}

In order to solve this equation we make use of the bisection method, and therefore need bounds on our initial guess for the pressure.

\subsection{Upper and Lower Bounds for Pressure}
We obtain a lower bound for the pressure with:
\begin{equation}
\tau=D\left(h\,W-1\right)-p\implies p=-\left(\tau+D\right)+D\,h\,W\geq-\left(\tau+D\right)+D\,h\,W\,\sqrt{v^{i}\,v_{i}}=-\left(\tau+D\right)+\sqrt{S^{i}\,S_{i}}.
\end{equation}
So, since the pressure must be non-negative, we have:
\begin{equation}
p\geq\text{MAX}\left[-\left(\tau+D\right)+\sqrt{S^{i}\,S_{i}},\text{SqrtTiny}\right].
\end{equation}

For an upper bound, we first note that:
\begin{equation}
h=1+\f{e+p}{\rho}=1+\f{\Gamma}{\Gamma-1}\f{p}{\rho}=1+\f{\Gamma}{\Gamma-1}\,\f{p\,W}{D},
\end{equation}
so,
%\begin{equation}
%    \tau=D\left(W+\f{\Gamma}{\Gamma-1}\f{p\,W^{2}}{D}-1\right)-p=D\left(W-1\right)+p\left(\f{\Gamma}{\Gamma-1}W^{2}-1\right).
%\end{equation}
%So,
%\begin{equation}
%    p=\f{\tau-D\left(W-1\right)}{\f{\Gamma}{\Gamma-1}W^{2}-1}.
%\end{equation}
%We also have:
%\begin{equation}
%    W=\left(1-v^{i}\,v_{i}\right)^{-1/2}=\left(1-\f{S^{i}\,S_{i}}{\left(\tau+D+p\right)^{2}}\right)^{-1/2}.
%\end{equation}
%Treating $p$ as an independent variable \sd{is this valid?}, we have:
%\begin{equation}
%    W\Big|_{p\rightarrow\infty}=1,
%\end{equation}
%which gives us an upper limit:
%\begin{equation}
%    p\leq\f{\Gamma-1}{\Gamma}\,\tau.
%\end{equation}
%Just to be safe, in the code we multiply this by two, so that:
%\begin{equation}
%    p\leq2\,\f{\Gamma-1}{\Gamma}\,\tau.
%\end{equation}
\begin{equation}
    \tau=D\left(W+\f{\Gamma}{\Gamma-1}\f{p\,W^{2}}{D}-1\right)-p=D\left(W-1\right)+p\left(\f{\Gamma}{\Gamma-1}W^{2}-1\right)>p\left(\f{\Gamma}{\Gamma-1}-1\right)=\f{p}{\Gamma-1}.
\end{equation}
So,
\begin{equation}
    p<\left(\Gamma-1\right)\tau.
\end{equation}
Typically, $\Gamma-1<3$. So, we end up with:
\begin{equation}
    p<3\,\tau.
\end{equation}